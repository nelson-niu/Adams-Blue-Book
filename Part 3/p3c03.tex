\documentclass[../main]{subfiles}
\begin{document}
\label{sec:p3ch03}
%johnDS
\chapter{Elementary properties of the category of CW-spectra}
%\chaptermark{\protect\parbox{0.7\linewidth}{Elementary properties of the category of CW-spectra}}


    
We want to show that CW-spectra can be manipulated very much
like CW-complexes. The standard way to make constructions for
CW-complexes is by induction over the cells. Now we can define ``stable
cells'' \index{stable cells} for CW-spectra. Let $C_n$ be the set of cells in $E_n$ other than
the base-point. Then we get a function 
\[C_n \lar{}C_{n+1} \text{ by } c_\alpha \mapsto \varepsilon_n(\Sigma c_\alpha)\]
This function is an injection. Let $C$ be the direct limit \[\lim_{n\to\infty} C_n;\]
an element of $C$ may be called a ``stable cell.'' Unwrapping the definition, a stable cell is an equivalence class of cells: for each $n$ such
an equivalence class contains at most one cell in $E_n$. Take two cells,
$c_\alpha$ in $E_n$ . and $c_\beta$ in $E_m$ and suppose without loss of generality $n\leq m$; then $c_\alpha$ and $c_\beta$ are equivalent if 
\[
\begin{tikzcd}
C_n \arrow[r] & C_{n+1} \arrow[r] & \dots \arrow[r] & C_m
\end{tikzcd}\]
maps $c_\alpha$ into $c_\beta$.

\begin{examples}
$E'\subset E$ is cofinal if and only if $C'\lar{} C$ is a bijection.

I said that the standard way to make constructions for CW-complexes is by induction over the cells, It is usual to order the cells of a
CW-complex by dimension: first we take the cells of dimension 0, then
the cells of dimension 1, and so on. For a CW-spectrum we can order
the stable cells by ``stable dimension'', but this ordering is not inductive
 in general, because we can have stable cells of arbitrarily large negative
stable dimension. Nevertheless we can perform inductive proofs, because each stable cell is attached to only a finite number of predecessors.
\end{examples}More formally, we have:
\begin{lemma}\label{lem:p3ch03.1}
let $E$ be a CW-spectrum, and G a subspectrum
of $E$ which is not cofinal. Then $E$ has a subspectrum $F$ such that
$E\supset F\supset G$ and $F$ contains just one more stable cell than $G$. 
\end{lemma}
\begin{proof}
$G$ is not cofinal, so there exists a stable cell $c$ in $E$
not in $G$. It has a representative $c_\alpha$,which is contained in a finite subcomplex $K\subset E_n$. So there exist finite subcomplexes $K$ containing representatives for stable cells in $E$ not in $G$. Among such $K$, choose
one with fewest cells. Let $K=L\cup e$, where $e$ is a top-dimensional
cell of $K$. Then $L$ fails to satisfy the conditions, for it has fewer cells
than $K$. So all the stable cells in $L$ represent stable cells in $G$. Then
there exists $m$ such that $\Sigma^mL$ gives a finite subcomplex of $G_{m+n}$
Form $F$ by adjoining $\Sigma^re$ to $G_{n+r}$ for $r>m$.
\end{proof}
We illustrate the use of this lemma by proving the homotopy extension theorem, Actually we prove something slightly more general. 
\begin{lemma}\label{lem:p3ch03.2}
Let $X,A$ be a pair of CW-spectra, and $Y,B$ a
pair of spectra such that $\pi_\ast(Y,B)= 0$. Suppose given a map $f:X\lar{}Y$
and a homotopy $h: \te{Cyl}(A)\lar{} Y$ from $f|_A$ to a map $g:A\lar{}B$. Then the homotopy can be extended over $\te{Cyl}(X)$ so as to deform $f$ to a map $X\lar{}B$.
\end{lemma}
The homotopy extension theorem is a special case when $B=Y$.
\begin{proof}
Work at the level of functions. Suppose $f$ is represented by a function $f':X'\lar{}Y$, and $h$ by a function $h':\te{Cyl}(A')\lar{}Y$, where $X'\supset A'$, $X'$ is cofinal in $X$ and $A'$ is cofinal in $A$. We make our induction using Zorn's Lemma. The objects to be ordered are pairs $(U,k')$ where $A'\subset U\subset X'$ and $k':\te{Cyl}(U)\lar{} Y$ is a function which deforms $f'|_U$ to a function into $B$. 
The set of such pairs is non-empty since $(A',h')$ qualifies; and it is clearly inductive. So we can choose a
maximal element $(U,k')$. I claim the maximal element has $U$ cofinal
in $X'$. If not, then by lemma \ref{lem:p3ch03.1} we can find $U\subset V\subset X'$ where $V$ contains just one more stable cell than $U$, say $V_n=U_n\cup e^m$. Then the maps \[f'_n|_{0\wedge e}: 0\wedge e\lar{} Y_{n-r} \]
\[k_n'|_{I^+ \wedge \partial e}: I^+ \wedge \partial e, 1\wedge \partial e\lar{} Y_{n-r},B_{n-r} \]
define an element of $\pi_m(Y_{n-r},B_{n-r})$. Now $\pi_\ast(Y,B)=0$, so that this element vanishes after sufficiently many suspensions.So on passing to $V_{n+p}= U_{n+p}\cup e^{m+p}$, we can extend $k'_{n+p}$ to a map \[k''_{n+p}: I^+\wedge e, 1\wedge e \lar{} Y_{p+n-r},B_{p+n-r}.. \]
Then define $k''_{n+q}$ for $q>p$ by suspension. This extension of $k'$ shows that $(U,k')$ is not maximal, a contradiction. This contradiction shows that $U$ is cofinal in $X'$, i.e., $U$ is cofinal in $X$. This gives the required map of $\te{Cyl}(X)$.
\end{proof}
A generalized version of Lemma \ref{lem:p3ch03.2} works when the inclusion $B\lar{}Y$ is replaced by a general function. 

\begin{customlemma}{$3.2'$} \label{lem:p3ch03.2'}
Let $X,A$ be a pair of CW-spectra and $\emptyset: B\lar{}  Y$ a function of spectra such that $\emptyset_\ast:\pi_\ast (B)\lar{}\pi_\ast(Y)$ is an isomorphism. Suppose given maps $f:X\lar{}Y$ and $g:A\lar{}B$ and a homotopy $h:\te{Cyl}(A)\lar{}Y$ from $f|_A$ to $\emptyset g$. Then we can extend $g$ over $X$ and $h$ over $\te{Cyl}(X)$ so that $h$ becomes a homotopy from $f$ to $\emptyset g:X\lar{}Y$.
\end{customlemma}
The proof is similar to that of $3.2$, except that we order triples $(U,k',g')$ where $g':U\lar{}B$ and $k' i_l = \emptyset g'$. The element \[k'_n|_{I^+\wedge \partial e}: I^+\wedge \partial e\lar{} Y_{n-r}\]
can be patched together with a contracting homotopy for $f|_{\partial e}$ to define an element of $\pi_m(\emptyset_{n-r}),$ say $k''_n$, which under the hypotheses must vanish on passing to $\emptyset_{p+n-r}$ for some $p$.

\begin{customlemma}{3.3} \label{lem:p3ch03.3}
Suppose that $\pi_\ast(Y)=0$, and $X,A$ is a pair of CW-spectra. Then any map $f:A\lar{}Y$ can be extended over $X$.
\end{customlemma}
\begin{proof}
Exercise. Either copy the proof of \ref{lem:p3ch03.2} or else quote the result of $\ref{lem:p3ch03.2}$.
\end{proof}

\begin{customthm}{3.4}\label{lem:p3ch03.4}
Let $f:E\lar{}F$ be a function between spectra such that $f_\ast:\pi_\ast(E)\lar{} \pi_\ast(F)$ is an isomorphism. Then for any CW-spectrum $X$, \[f_\ast:[X,E]_\ast \lar{}[X,F]_\ast\] is a (1-1) correspondence.
\end{customthm}
I emphasize that $E$ and $F$ are not assumed to be CW-spectra.
By analogy with the case of CW-complexes, a function $f: E\lar{} F$
between spectra such that $ [X,E]_\ast \lar{f_\ast}[X,F]_\ast$ correspondence
for all CW-spectra X would be called a weak equivalence. 

\begin{proof}
(First argument). Without loss of generality we
can suppose that $f$ is an inclusion; for if not, replace $F$ by the
spectrum $M$ in which $M_n$ is the mapping cylinder of $f_n$, Then
$\pi_\ast(F,E) = 0$ by the exact sequence. Now we see that $f_\ast$ is an epimorphism by applying lemma \ref{lem:p3ch03.2}, taking the pair $X$ mod $A$ to be $X$ mod $\te{pt.}$
Similarly, we see that $f_\ast$ is a monomorphism by applying lemma \ref{lem:p3ch03.2}, taking the pair $X$ mod $A$ to be $\te{Cyl}(X)$ mod its ends.

(Second argument). Instead of using the mapping cylinder spectrum,
use Lemma \ref{lem:p3ch03.2'} in the above argument. 
\end{proof}

\begin{customcor}{3.5}[Compare the theorem of J.H.C. Whitehead.] \label{cor:p3ch03.5} Let $f:E\lar{}F$ be a morphism between CW-spectra such that:
\[f_\ast:\pi_\ast(E)\lar{} \pi_\ast(F)\]
is an isomorphism. Then $f$ is an equivalence in our category. 
\end{customcor}
The deduction of \ref{cor:p3ch03.5} from \ref{lem:p3ch03.4} is a triviality, valid in any category. 

\begin{examples}
Let $f: E\lar{} F$ be a function such that $f_n: E_n\lar{} F_n$ is a
homotopy equivalence for each $n$. Then $f$ is an equivalence in our
category. 
\end{examples}
\begin{exercise}
Use (\ref{cor:p3ch03.5}) to show that any CW-spectrum $Y$ is equivalent in our category to an $\Omega_0$-spectrum.
\end{exercise}
\begin{hint}
Construct a functor $T_n$ from CW-complexes to spectra by \[(T_nX)_r =\begin{cases}\Sigma^{r-n}X & r\geq n\\ \te{pt.} & r<n\end{cases}\]
Form the set of morphisms in our category \[[T_nX,Y]_0,\] and check that it is a representable functor, represented say by $Z_n$. Observe that the $Z_n$ give the components of an $\Omega_0$-spectrum $Z$; construct a function $Y\lar{}Z$ and apply \ref{cor:p3ch03.5}.
\end{hint}

Now I must reveal that we would really like a relative form of the
theorem of J.H.C. Whitehead. If $X$ is a spectrum, let $\te{Cone}(X)$ be
the spectrum whose $n\nth$ term is $I\wedge X_n$, with maps
\[\begin{tikzcd}
(I\wedge X_n)\wedge S^1 \arrow[rr, "1\wedge \varepsilon_n"] &  & I\wedge X_{n+1} 
\end{tikzcd} \text{(We take the base-point in I to be 0.)}\] 
We have an obvious inclusion function $i:X\lar{} \te{Cone}(X)$ (use the end of the cone).
\begin{customthm}{3.6}\label{thm:p3ch03.6}
Let $f:E,A\lar{} F,B$ be a function between pairs of spectra such that \[f_\ast:\pi_\ast(E,A)\lar{} \pi_\ast (F,B)\] is an isomorphism. Then for any CW-spectrum X, \[f_\ast:[\te{Cone}(X),X; E,A]_\ast \lar{}[\te{Cone}(X),X; F,B]_\ast\]
is a 1-1 correspondence.
\end{customthm}
\begin{proof}[Proof Sketch]
Construct a new spectrum $R$ (for relative) with $R_n=L(E_n,A_n)$ (the space of paths in $E_n$ starting at the base-point and finishing in $A_n$) and structure maps $\rho_n$ given by 
\[\begin{tikzcd}
{L(E_n,A_n)} \arrow[rr, "L\varepsilon'_n"] &  & {L(E_{n+1} \Omega, A_{n+1} \Omega)\cong (L(E_{n+1},A_{n+1}))\Omega}
\end{tikzcd}  \]
where the $\Omega$ is written on the right to keep the "loops" coordinates out of the way of the path coordinate. Similarly, construct $S$ (not, for the moment, the sphere spectrum) with $S_n=L(F_n,B_n)$. Then $f$ induces a function of spectra $R\lar{} S$, inducing an isomorphism of absolute homotopy groups. By theorem \ref{lem:p3ch03.4}, 
\[[X,R]_\ast \lar{} [X,S]_\star \]
is a 1-1 correspondence. Unwrapping this, it says \[f_\ast:[\te{Cone}(X),X; E,A]_\ast \lar{}[\te{Cone}(X),X; F,B]_\ast\]
is a 1-1 correspondence.
\end{proof}
This application shows why I specified that $E$ and $F$ in \ref{lem:p3ch03.4} need
not be CW-spectra. 

Now for any spectrum $X$, we will define $\te{Susp}(X)$ so that its $n\nth$ term is $S^1\wedge X_n$ and its structure maps are 
\[\begin{tikzcd} (S^1\wedge X_n) \wedge S^1 \arrow[rr,"1\wedge \xi_n"]& &S^1\wedge X_{n+1}\end{tikzcd}.\] $\te{Susp}$ is obviously a functor.

\begin{customthm}{3.7}\label{thm:p3ch03.7}
$\te{Susp}:[X,Y]_\ast\lar{} [\te{Susp}(X),\te{Susp}(Y)]_\ast$ is a $1-1$ correspondence.
\end{customthm}
This theorem assures us that in some sense we did succeed in getting into a stable situation.
\begin{proof}
We have the following commutative diagram.
\[\begin{tikzcd}
{[X,Y]_\ast} \arrow[d, "\te{Susp}"] \arrow[rr, "\te{Cone}"] &  & {[\te{Cone}(X),X; \te{Susp}(Y),Y]_\ast} \arrow[d, "j_\ast"] \\
{[\te{Susp}(X),\te{Susp}(Y)]_\ast} \arrow[rr, "j^\ast"]     &  & {[\te{Cone}(X),X; \te{Susp}(Y),\te{pt.}]_\ast}             
\end{tikzcd}\]
Now the map $\te{Cone}$ is clearly injective (since restriction gives an inverse for it) and surjective (by Lemma \ref{lem:p3ch03.3}). Also $j^\ast$ is clearly a 1-1
correspondence. The proof will be complete as soon as we show that $j_\ast$
is a 1-1 correspondence, by quoting Theorem \ref{thm:p3ch03.6} and proving: 

\begin{customlemma}{3.8} \label{lem:p3ch03.8}
$j_\ast : \pi_\ast(\te{Cone}(Y),Y)\lar{} \pi_\ast(\te{Susp}(Y),\te{pt.})$ is a 1-1 correspondence. 
\end{customlemma}


Consider the following commutative diagram:

\[
\adjustbox{scale=0.9,center}{
\begin{tikzcd}
{\pi_{n+r+1}(I\wedge Y_n,Y_n)} \arrow[swap]{r}{\cong}[swap]{\partial} \arrow[dd] \arrow[rrrd, "j_\ast", bend left=35] &   \pi_{n+r}(Y_n) \arrow[d]                                                                   &  &                                                                       \\
                                                                                                        & \pi_{n+r+1} (Y_n \wedge S^1) \arrow[d, "(\eta_n)_\ast"] \arrow[rr, "(-1)^{n+r} \tau_\ast"] &  & \pi_{n+r+1}(S^1\wedge Y_n) \arrow[d]                                  \\
{\pi_{n+r+2}(I\wedge Y_{n+1},Y_{n+1})} \arrow[swap]{r}{\cong}[swap]{\partial} \arrow[rrrd, "j_\ast", bend right=35]    & \pi_{n+r+1} (Y_{n+1}) \arrow[d]                                                            &  & \pi_{n+r+2}(S^1\wedge Y\wedge S^1) \arrow[d, "(1\wedge \eta_n)_\ast"] \\
                                                                                                        & \pi_{n+r+2} (Y_{n+1} \wedge S^1) \arrow[rr, "(-1)^{n+r+1} \tau_\ast"]                      &  & \pi_{n+r+2}(S^1\wedge Y_n)                                           
\end{tikzcd}
}\]
$\pi_\ast(\te{Cone}(Y),Y)$ is the direct limit of the left-hand column, and the
diagram shows it is isomorphic to $\displaystyle \lim_{n\to \infty} \pi_{n+r}(Y_n).$ $\pi_\ast(\te{Susp}(Y),\te{pt.})$ is
the direct limit of the right-hand column, and the diagram shows that it
ig isomorphic to the direct limit of the system in which the groups are $\pi_{n+r+1}(Y_n\wedge S^1)$ and the maps are the vertical arrows in the center
column. But the center column shows that these two direct limits are the
same, This proves Lemma \ref{lem:p3ch03.8}, which proves Theorem \ref{thm:p3ch03.7}.
\end{proof}
Now we can remark that $[\te{Susp}(X), Z]$ is obviously a group, because in $\te{Susp}(X)$ we have a spare suspension coordinate out in front to manipulate. And for the same reason, $[\te{Susp}^2(X), Z]$ is an abelian
group. But now we can give $[X, Y]$ the structure of an abelian group,
because $[X,Y]$ is in 1-1 correspondence with $[ \te{Susp}^2(X), \te{Susp}^2(Y)]$,
and we pull back the group structure on that. So now our sets of
morphisms $[X,Y]$ are abelian groups, and it's easy to see that composition is bilinear. 

Actually there is a unique way to give each set of morphisms $[X,Y]$
the structure of an abelian group so that composition is bilinear; this is
standard once I've said the usual categorical things about sums and
products.

Well, now I would like to say that I have an additive category. The
existence of a trivial object is easy: we take the spectrum $E_n=\te{pt.}$ for all $n$. Then $[X,\te{pt.}]=0$ and $[\te{pt.},X]=0 $.

I claim this category has arbitrary sums (= coproducts). In fact,
given spectra $X_\alpha$ for $\alpha\in A$, we form $X=\bigvee_\alpha X_\alpha$ by $X_n=\bigvee (X_\alpha)_n$ with the obvious structure maps.

\begin{tikzcd}
	{X_n\wedge S^1 = \left(\bigvee_\alpha (X_\alpha)_n\right)\wedge S^1 = \bigvee_\alpha (X_\alpha)\wedge S^1} && {\bigvee_\alpha (X_\alpha)_{n+1}}
	\arrow["{\bigvee_\alpha \xi_{\alpha n}}", from=1-1, to=1-3]
\end{tikzcd}

This obviously has the required property:
\[\bigg[\bigvee_\alpha X_\alpha,Y\bigg] \lar{\cong} \prod_\alpha [X_\alpha,Y] \]

Now I must talk about cofiberings. Suppose given a map $f:X\lar{}Y$ between CW-spectra. It is represented by a function $f':X'\lar{}Y$, where $X'$ is a cofinal subspectrum. Without loss of generality I can suppose $f'$ is cellular, i.e., $f'_n$ is a cellular map of CW-complexes for each $n$. We form the mapping cone $Y\cup_{f'} CX$ as follows: its $n\nth$ terms is $Y_n \cup_{f_n'} (I\wedge X_n')$ and the structure maps are the obvious ones. If we replace $X'$ by a smaller cofinal subspectrum $X''$, we get $Y\cup_{f''} CX''$ which is smaller than $Y\cup_{f'} CX'$, but cofinal in it, and so equivalent. So the construct depends essentially only on the map $f$, and we can write it $Y\cup_f CX$. If we vary $f$ by a homotopy, $Y\cup_{f_0} CX$ and $Y\cup_{f_1} CX$ are equivalent, but the equivalence depends on the choice of homotopy.

Let $X$ be a CW-spectrum, $A$ a subspectrum. I will say $A$ is a \emph{closed} \index{closed subspectrum} if for every finite subcomplex $K\subset X_n$, $\Sigma^mK\subset A_{m+n}$ implies $K\subset A_n$. That is, if a cell gets into $A$ later, I put it into $A$ to start with. It is equivalent to saying that $A\subset B\subset X$, $A$ cofinal in $B$ implies that $A=B$.

Suppose that $i:X\lar{}Y$ is the inclusion of a closed subspectrum. Then we can form $Y/X$, with the $n\nth$ term $Y_n/X_n$. In this case there is a map \[r:Y\cup_i CX\lar{} Y/X \]
with components \[Y_n\cup_{i_n} CX_n \lar{} Y_n/X_n \]
The map $r$ is an equivalence, by corollary \ref{cor:p3ch03.5}.

Let's return to the general case. We have morphisms \[X\lar{f}Y\lar{i} Y\cup_f CX\]

\begin{customprop}{3.9}\label{prop:p3ch03.9}
For each $Z$ the sequence \[[X,Z]\lal{f^\ast} [Y,Z]\lal{i^\ast} [Y\cup_f CX,Z]\]
is exact
\end{customprop}
The proof is the same as for CW-complexes, and is trivial, because homotopies were defined in terms of maps of cylinders.

The sequence $X\lar{f}Y\lar{i} Y\cup_f CX$, or anything equivalent to it, is called a \emph{cofibre sequence} or \emph{Puppe sequence} \index{cofibre sequence} \index{Puppe sequence}. We can extend cofiberings to the right, by taking 
\[X\lar{f} Y\lar{i} Y\cup_f CX \lar{} (Y\cup_f CX)\cup_i CY \]
The last spectrum is equivalent to $(Y\cup_f CX)/Y = \te{Susp}(X)$. If we continue the sequence further, we get
\[\begin{tikzcd}
	X & Y & {Y\cup_f CX} & {\te{Susp}(X)} && {\te{Susp}(Y)}
	\arrow["f", from=1-1, to=1-2]
	\arrow["i", from=1-2, to=1-3]
	\arrow["j", from=1-3, to=1-4]
	\arrow["{-\te{Susp}(f)}", from=1-4, to=1-6]
\end{tikzcd}\]
as for CW-complexes. It follows that the exact sequence of Proposition \ref{prop:p3ch03.9} can also be extended to the right.

\begin{customprop}{3.10}\label{prop:p3ch03.10}
The sequence \[[W,X]\lar{f_\ast} [W,Y] \lar{i_\ast} [W,Y\cup_f CX] \]
is exact.
\end{customprop}
In other words, in our category cofiberings are the same as fibering.
\begin{proof}
Since $if \sim 0$, $i_\star f_\star=0$. Suppose given $g:W\lar{} Y$ such that $ig\sim 0$. Then we can construct the following diagram of cofiberings.

\[
\adjustbox{scale=1, center}{
\begin{tikzcd}
	X & Y & {Y\cup_f CX} & {\te{Susp}(X)} && {\te{Susp}(Y)} \\
	W & W & CW & {\te{Susp}(W)} && {\te{Susp}(W)}
	\arrow["f", from=1-1, to=1-2]
	\arrow["i", from=1-2, to=1-3]
	\arrow["j", from=1-3, to=1-4]
	\arrow["{-\te{Susp}(f)}", from=1-4, to=1-6]
	\arrow["1", from=2-1, to=2-2]
	\arrow["i", from=2-2, to=2-3]
	\arrow["j", from=2-3, to=2-4]
	\arrow["{-1}", from=2-4, to=2-6]
	\arrow["g"', from=2-2, to=1-2]
	\arrow["h"', from=2-3, to=1-3]
	\arrow["k"', from=2-4, to=1-4]
	\arrow["{\te{Susp}(g)}"', from=2-6, to=1-6]
\end{tikzcd}}\]

(The homotopy $ig\sim 0$ gives us $h$, and the rest follows automatically.)

Now by Theorem \ref{thm:p3ch03.7} we have $k=\te{Susp}(\ell)$ for some $\ell \in [W,X]$, and \[\left(-\te{Susp}(f)\right) \left(\te{Susp}(\ell)\right) \simeq \left(\te{Susp}(g)\right)(-1) \]
i.e.,
\[\te{Susp}(f\ell) \simeq \te{Susp}(g)\]
so using Theorem \ref{thm:p3ch03.7} again, we have $f\ell \simeq g$. This proves Proposition \ref{prop:p3ch03.10}.
\end{proof}
\begin{customprop}{3.11}\label{prop:p3ch03.11}
Finite sums are products.
\end{customprop}
In fact, \[X\lar{} X\vee Y \lar{} Y \]
is clearly a cofibering, because $(X\vee Y)\cup CX \simeq Y$. So by \ref{prop:p3ch03.10}

\[[W,X] \lar{} [W,X\vee Y] \lar{} [W,Y] \]
is exact; but it is clearly split short exact, so \[[W,X\vee Y] \cong [W,X]\oplus [W,Y] \] and $X\vee Y$ is also the product $X$ and $Y$.

Now I know that my category in an additive category.

\begin{customthm}{3.12}\label{thm:p3ch03.12}
The Representability Theorem of E.H. Brown is valid in the category of CW-spectra and morphisms of degree $0$.
\end{customthm}
The proof is as usual, but arrange the induction right.

\begin{exercise}
Use \ref{thm:p3ch03.12} to show that any spectrum $Y$ is weakly equivalent to a CW-spectrum. (Consider the functor $[X,Y]_0$.)
\end{exercise}
\begin{customprop}{3.13}\label{prop:p3ch03.13}
The stable category has arbitrary products.
\end{customprop}
\begin{proof}
The functor of $X$ given by \[\prod_\alpha [X,Y_\alpha]_0 \] satisfies the data of Brown's theorem, so it is representable, Now we
see that this representing object works for maps of degree $r$ as well.
\end{proof}

Note next that for any collection of $X_\alpha$ we have a morphism \[\bigvee_\alpha X_\alpha \lar{} \prod_\alpha X_\alpha\]
Namely, for each $\alpha$ and $\beta$ I have to give a component which is a map $X_\alpha \lar{} X_\beta$; I take it to be $1$ if $\alpha=\beta$, $0$ if $\alpha\neq \beta$.

\begin{customprop}{3.14}\label{prop:p3ch03.14}
(This form is due to Boardman). Suppose that for each $n$, $\pi_n(X_\alpha) =0$ for all but a finite number of $\alpha$. Then the map  \[\bigvee_\alpha X_\alpha \lar{} \prod_\alpha X_\alpha\]
is an equivalence.
\end{customprop}
\begin{proof}
First note that \[\pi_n(X_1\vee X_2) \cong \pi_n(X_1)\oplus \pi_n(X_2)\]
under the obvious maps. (see \ref{prop:p3ch03.11}) \begin{exercise}
Prove this directly from the definitions of $\pi_\ast$ and $X_1\vee X_2$.
\end{exercise}
By induction, we have \[\pi_n(X_1\vee \dots \vee X_m) \cong \sum_{i=1}^m \pi_n(X_i) \]
for finite wedges. Now we have \[\pi_n\bigg(\bigvee_\alpha X_\alpha\bigg) = \sum_\alpha \pi_n(X_\alpha)\]
by passing to direct limits. Also \[\pi_n\bigg(\prod_\alpha X_\alpha\bigg)= \prod_\alpha \pi_n (X_\alpha), \text{ by definition.} \]
Now the data was chosen precisely so that $\displaystyle \sum_\alpha \pi_n (X_\alpha) \lar{} \prod_\alpha \pi_n(X_\alpha) $ is an isomorphism. Therefore $\displaystyle \bigvee_\alpha X_\alpha \lar{} \prod_\alpha X_\alpha$ is an equivalence, by \ref{cor:p3ch03.5}.
\end{proof}

\begin{remark*}
If we use the direct proof that \[\pi_n(X_1\vee X_2)\cong \pi_n(X_1)\oplus \pi_n(X_2)\]
this gives a proof that finite sums are products, independently of \ref{thm:p3ch03.7},
but depending on Brown's theorem. This can be used, in a way which is
familiar to categorists, to define an addition in the sets $[X, Y]$; this
way of introducing the addition is independent of \ref{thm:p3ch03.7}. Of course you have
to show that the addition makes the sets $[X,Y]$ into abelian groups; the
main point is to establish the existence of inverses. I recommend making
use of an argument which is standard for H-spaces, as follows. Since
$X\vee X$ is both a sum and product, you can make a map

\[X\vee X\lar{} X\vee X\]
with components $\begin{bmatrix}1&1\\0&1\end{bmatrix}$. Check that it satisfies the hypotheses of \ref{cor:p3ch03.5}, so it has an inverse. The inverse has the form $\begin{bmatrix}1&\nu\\0&1\end{bmatrix}$. But you know the inverse of $\begin{bmatrix}1&1\\0&1\end{bmatrix}$ is $\begin{bmatrix}1&-1\\0&1\end{bmatrix}$; so you use $\nu$ for inversion and it works.
\end{remark*}

\end{document}